\documentclass[11pt, a4paper]{jarticle}

\usepackage{FIXfuninfosys}
\usepackage{url}
\usepackage[dvipdfmx]{graphicx}
\usepackage[labelsep=quad]{caption}
\usepackage{float}

\author{% 
b10xxxxx 未来太郎\\指導教員 : 函館一郎
}
%\course{Information Systems Course}
%\course{Advanced ICT Course} %% 高度ICTコースの場合はこちらを使用
%\course{Information Design Course} %% 情報デザインコースの場合はこちらを使用
\course{Complex Systems Course} %% 複雑系科学コースの場合はこちらを使用
%\course{Intelligent Systems Course} %% 知能システムコースの場合はこちらを使用

\title{中間報告書の書き方}
\etitle{How to Write Manuscripts for Midterm Report}
\eauthor{Taro MIRAI}

\abstract{和文は300から400文字で記述すること.和文は300から400文字で記述すること.和文は300から400文字で記述すること.和文は300から400文字で記述すること.和文は300から400文字で記述すること.和文は300から400文字で記述すること.和文は300から400文字で記述すること.和文は300から400文字で記述すること.和文は300から400文字で記述すること.和文は300から400文字で記述すること.和文は300から400文字で記述すること.和文は300から400文字で記述すること.}

\keywords{北海道, 函館, 亀田中野, 公立はこだて未来大学}

\eabstract{English should be written in 100 to 150 words.English should be written in 100 to 150 words.English should be written in 100 to 150 words.English should be written in 100 to 150 words.English should be written in 100 to 150 words.English should be written in 100 to 150 words.English should be written in 100 to 150 words.English should be written in 100 to 150 words.English should be written in 100 to 150 words.English should be written in 100 to 150 words.English should be written in 100 to 150 words.}

\ekeywords{Hokkaido, Hakodate, Kamedanakano, FUN}

\begin{document}
\maketitle
%\vspace*{-.5cm}

\section{背景と目的}

このサンプルは情報システムコースにおける中間報告書の様式について説明したものである.必ずしもこの雛形を使う必要はないが,仕上がりイメージはできる限りこの雛形にあわせること.

用紙サイズはA4,向きは縦とし,上下の余白は30mm、左右の余白は25mmとする.本文には明朝体とTimes New Romanを用いる.ただし,タイトルや章節の見出し,図表のキャプションはゴシック体とする.タイトルは14ポイント,氏名と章の見出しは12ポイント,節の見出しは11ポイント,その他は10ポイントとする.また,和文タイトルから英文キーワードまでは1段,本文は2段で構成とし,1段のセクションは42文字×45行,2段のセクションは20文字×45行とする.

なお,章立てはあくまでも参考であり,これに限らない.

\section{関連研究}

中間報告書の文量は4ページとする.学籍番号をファイル名としたPDFファイル1つにまとめた形で作成すること.提出するファイル名はb10xxxxx.pdfとする.

句読点は「,」,「.」とする.「、」,「。」は使用しない.アブストラクトなど英文表記の部分については,スペルチェックプログラムによるチェックをする.

\section{提案する理論}

\subsection{数式}

数式による記述が必要な場合は,式番号を式(\ref{eq1})のように適切に参照しながらまとめること.
\begin{equation}
  \label{eq1}
  f(x) = \frac{1}{2 \phi} \exp \left( - \frac{ | x - \mu | }{\phi}  \right)
\end{equation}
式はequationやeqnarrayを用いて記述する.式(\ref{eq2})から(\ref{eq4})はeqnarryの例である.
\begin{eqnarray}
  \label{eq2}
  2x_1 + x_2 & = & 5 \\
  \label{eq3}
  2x_2 & = & 1 \\
  \label{eq4}
  x_1 & = & \frac{9}{4}
\end{eqnarray}



\subsection{図・写真}

読者の理解を助けるため,図や表を効果的に利用すること.図\ref{fig1}のキャプションは下記のようにする.

\begin{figure}[H]%Hで半ば強制的にそこに配置することができる。t(top)やb(bottom)の指定もある
  \includegraphics[scale=0.5]{hoge.png}
  \caption{画像のテスト}
  \label{fig1}%labelを付けておくと、本文中で引っ張れて便利
\end{figure}

このように,図の下に記すこと.表\ref{tab1}のキャプションは下記のようにする.

\begin{table}[H]
  \caption{表のテスト}
  \begin{center}
  \begin{tabular}{ r  c  p{8zw} } %縦罫線は引かないらしい。r(右寄せ), c(中央寄せ), l(左寄せ)で指定できる
    \hline
    項目1 & 項目2 & 項目3 \\
    \hline
    a & b & c \\
    \hline
  \end{tabular}%tabularは閉じること
  \end{center}
  \label{tab1}
\end{table}

このように,表の上に記す.

\section{実験と評価}
参考文献はbibにまとめておきました.取得してくる場合には\cite{marumaru}\cite{abc}\cite{batubatu}\cite{efghij}みたいにすれば取得できます.

\section{考察}

\section{結言}

\bibliographystyle{unsrt}%unsrt(引用された順)と、plain(アルファベット順)の二種類のソートがある。ほかにもある
\bibliography{bib}%bibは手打ちするもんじゃねぇ…(体感)


\end{document}
%
%
% EOF 
